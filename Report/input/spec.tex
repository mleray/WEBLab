\section{Specification}
\label{sec:spec}

\subsection{Needed informations}
\label{ssec:info} 

In order to calculate the user's blood alcohol concentration (BAC), we need a few informations about him and about his drinking. More precisely, we need the elements listed below:

\begin{itemize}
\item Gender (M/F);
\item Weight;
\item Number of hours since the drinking began; 
\item Amount of alcohol.
\end{itemize}

\subsection{BAC calculation formula}
\label{ssec:formula}

To calculate the blood alcohol concentration, we will use the Widmark formula, which is the following:

\begin{equation}
   BAC = A \div (R \times M) - 0.015 \times H
\end{equation}
   
In this formula, we have:

\begin{itemize} 
\item BAC : Blood Alcohol Concentration;
\item A : Alcohol ingested, in grams;
\item R : Ratio (0.70 for men, 0.55 for women);
\item M : Body weight, in kilograms;
\item H : Number of hours since the drinking began.
\end{itemize}

To obtain the value of A, we use this formula:

\begin{equation}
   A = (V \times P \times 0.8) \div 100
\end{equation}

In this formula, we have:

\begin{itemize}
\item V : Volume of alcohol in milliliters;
\item P : Alcohol degree, in percentage.
\end{itemize}

\subsection{Elimination of alcohol}
\label{ssec:elim}

The median rate of decrease in BAC is considered to be 15 milligrams per cent (mg\%) per hour. In our calculations, we used the approximations listed in {\sc table}~\ref{tab:elim}.

\begin{table}[H]
\centering
\begin{tabular}{|c|c|}
  \hline
  \bf{BAC value} & \bf{Hours until sober}\\
  \hline
	0.016 & 1\\
	0.05 & 3.75\\
	0.08 & 5\\
	0.10 & 6.25\\
	0.16 & 10\\
	0.20 & 12.5\\
	0.24 & 15\\
  \hline
\end{tabular}
\caption{Hours until sober depending on the BAC value}
\label{tab:elim}
\end{table}