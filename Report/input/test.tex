\section{Testing instructions}
\label{sec:test}

When you get to Beerculator's web page, the first thing to do is to enter your personal informations. To do so, in the \guillemotleft{} User data \guillemotright{} tab ({\sc figure}~\ref{fig:userData}), you have to indicate your weight in kilogramms and then your gender (Male/Female). You should also indicate the time when you started drinking. Once you are done here, you should click on the \guillemotleft{} Save \guillemotright{} button.\\

\begin{figure}[H]
	\centering
   \includegraphics[scale=0.65]{./figures/userData.jpg}
   \caption{User data tab in the user interface of Beerculator}
   \label{fig:userData}
\end{figure}

After that, you can move on to the right side of the screen and start adding drinks to your list, in the \guillemotleft{} Drinks \guillemotright{} tab ({\sc figure}~\ref{fig:drinks}). To do so, you can click on the green squares corresponding to the drinks you have had. If you made a mistake, you can click on the associated red square in order to cancel and remove this drink. If you update your drinks this way, they are automatically saved. But you can also edit the quantity of each drink directly in the corresponding text box, which allows you to add or remove several drinks at the same time. If you do so, you should click on the \guillemotleft{} Save \guillemotright{} button.\\

\begin{figure}[H]
	\centering
   \includegraphics[scale=0.65]{./figures/drinks_tab.jpg}
   \caption{Drinks tab in the user interface of Beerculator}
   \label{fig:drinks}
\end{figure}

Then, you can launch the computation in the \guillemotleft{} Calculation \guillemotright{} tab ({\sc figure}~\ref{fig:calcul}) by clicking on the \guillemotleft{} Calculate \guillemotright{} button. The BAC value is indicated in the first text block, \guillemotleft{} Alcohol level \guillemotright{}, and it is followed by when you will be sober again (text block \guillemotleft{} Sober at \guillemotright{}).

\begin{figure}[H]
	\centering
   \includegraphics[scale=0.65]{./figures/calcul.jpg}
   \caption{Calculation tab in the user interface of Beerculator}
   \label{fig:calcul}
\end{figure}